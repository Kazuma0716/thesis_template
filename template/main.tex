\documentclass[a4paper,11pt]{jsbook}

\newcommand{\V}[1]{\boldsymbol{#1}}
\def\thline{\noalign{\hrule height 1pt}}
\def\tvline{\vrule width 1pt}

\usepackage{here} %図の場所の指定で[H](ここに貼る)を指定するためのパッケージ
\usepackage{makeidx}
\usepackage{amsmath}
\usepackage{amssymb}
\usepackage[dvipdfmx]{graphicx} %dvipdfmxはjpgやpngの張り込みのために使用

\makeindex

\pagenumbering{roman}

\begin{document}
% 表紙
\title{令和元年度 卒業論文\\
ドリルキングアンセムの研究}

\author{鳴海 和真 \\
Chiba Institute of Technology}

\date{2017年2月x日}

\maketitle

%%% 但し書き等 %%%
この論文は、読んだあと自動的に消滅する。
\clearpage

%%% 献辞 %%%
% D論、あるいは誰かを亡くしたときの卒論、修論等で
% 配偶者や配偶者の予定となる人の名前は覚悟をもって書くこと
\thispagestyle{empty}
\vfil
\ \\
\vspace{15em}
\begin{center}
	{\Large 最愛の京成線に捧ぐ }
\end{center}
% 献辞をかかない場合はここまでコメントアウト

\chapter*{謝辞}\addcontentsline{toc}{chapter}{謝辞}

 本研究を進めるにあたり、ご指導を頂いた卒業論文指導教員の上田隆一准教授に感謝いたします。


\tableofcontents

%\cleardoublepage

%%% 本文 %%%
% 章のページの先頭は左側(奇数ページ)に来る

\cleardoublepage
\pagenumbering{arabic}

\chapter{序論}

\section{背景}

Simultaneous Localization and Mapping(以下, SLAM)は、自己位置推定と環境地図の生成を同時に行う技術である。SLAMは、工場で稼働する無人運搬車や自動車の自動運転、ロボット掃除機といった自律行動をするロボットに活用されている\cite{Kondo2019}。中でも、カメラ映像からSLAMを行うものをVisual SLAMという。単眼カメラSLAMはVisual SLAMを単眼カメラのみで行う。単眼カメラSLAMはVisual SLAMの中で最も安価かつ小型で消費電力を抑えられることが長所である\cite{Artal2017}。
\section{問題}

単眼カメラSLAMには、空間把握をカメラ画像に依存するために、エッジや模様のない環境に対して特徴点検出ができなくなるという問題が存在する。このような場合にはSLAMが中断される。この問題の解決策として、IMUやRGB-Dといったセンサを併用することによる方法が一般的である。しかし、この方法ではセンサの高価格化、大型化および消費電力の増加によって前述した単眼カメラSLAMの長所を活かすことができない。

\section{目的}

単眼カメラSLAMを中断されることなく完遂することのできる環境要件を調査することを目的とする。ここでの環境要件とは単眼カメラSLAMが中断されないために最低限必要とされる特徴とする。本目的の達成によってロボットに用いられるセンサのコスト低下、軽量化されることによりSLAM導入のハードルを低下させることができると言える。

\section{本論文の構成}

本論文では、2章で提案手法を述べ、3章で実験の方法と結果を示し、4章で結論を述べる。

% dvipdfmxとhereのテスト
%\begin{figure}[H]
%	\begin{center}
%		\includegraphics[width=1.0\linewidth]{../zero.png}
%		\caption{}
%		\label{fig:}
%	\end{center}
%\end{figure}
%

\include{purpose}
\chapter{提案手法}\label{chap:method}

\section{問題設定}

今回は、倉庫等特定の環境で稼働するロボットを想定する。予め単眼カメラSLAMを行うロボットの補助のために環境改善が可能であることとする。ここで、環境への影響を最小限に抑えて単眼カメラSLAMが中断する環境を改善する条件を実験により調査する。

\section{解決方法}

単眼カメラSLAMによる特徴点検出ができない環境にマーカを設置することで問題を解決する。このとき、環境への影響が最小限になるようなマーカの範囲を求める。

\chapter{結論}

単眼カメラSLAMを完遂するために必要な特徴点マッチング数を絞り込む手法を提案した。同様の手順を行うことで単眼カメラSLAMを完遂できない環境を最小限のマーキングによって改善可能である。
課題点として、本研究の実験方法はトラッキングが開始されることを前提にしている点が挙げられる。実験において棚に該当する特徴点検出が容易な物体を隣接していない環境では同じ手法では不十分である。この課題の解決には新たな実験条件を考案する必要がある。




\appendix
\include{appendix}

%%% 参考文献 %%%
% よほどのことが無い限りet al.は使わないことにしましょう
\bibliographystyle{jualpha}
\bibliography{./references}

\newpage
\printindex

\end{document}


